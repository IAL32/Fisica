\begin{figure}[h!]
    \textbf{Tema d'Esame di Gennaio 2017}\\ \\
    Si ha la necessità di far fuoriuscire dell'acqua (densità $1000 kg/m^3$
    ) contenuta all'interno di una siringa senza ago, posta in orizzontale, alla velocità di $15 cm/s$.
     Stabilire quale differenza di pressione bisogna esercitare tra lo stantuffo e il beccuccio da cui fuoriesce il
    fluido, sapendo che il rapporto tra le due sezioni vale $20$. 
\end{figure}

\begin{figure}[h!]
    \textbf{Tema d'Esame di Febbraio 2017}\\ \\
    Un torchio idraulico è costituito da due vasi cilindrici comunicanti tra loro e contenenti acqua, disposti verticalmente, di sezioni $S_A =2 dm^2$  e $S_B = 10 dm^2$ , rispettivamente. Dentro i vasi possono scorrere, a tenuta e senza attrito, due pistoni $A$ e $B$ di masse $m_A = 20 kg$ e $m_B = 150 kg$. Si calcoli la massa m del carico che si deve porre sul pistone A per ottenere livelli uguali nei due vasi.
\end{figure}

\begin{figure}[h!]
    \textbf{Tema d'Esame di Giugno 2017}\\ \\
    L’impianto idrico di una casa ha una pompa in un pozzo con un tubo di uscita con raggio interno di $6.3 mm$. Si assuma che la pompa possa mantenere una pressione relativa di $410 kPa$ nel tubo di uscita. Una doccia posta $6.7 m$ più in alto della pompa ha un erogatore con $36$ fori, ciascuno di raggio $0.33 mm$. Se il rubinetto della doccia è completamente aperto, a quale velocità esce l’acqua dalla doccia?
\end{figure}

\begin{figure}[h!]
    \textbf{Tema d'Esame di Luglio 2017}\\ \\
    Una bottiglia (volume $V_B = 2.0 L$ e massa $80 g$) contiene $40$atm di $He$ (gas perfetto) a temperatura ambiente. Calcolare la forza che bisogna esercitare verticalmente sulla bottiglia per tenerla completamente immersa in acqua. (Massa atomica $He = 6,64\cdot 10^{24} g$. Densità dell’acqua $\rho = 1000 kg/m^3$)
\end{figure}

\begin{figure}[h!]
    \textbf{Tema d'Esame di Settembre 2017}\\ \\
    Si ha una siringa piena di acqua (densità $1000 kg/m^3$
). Il rapporto tra le sezioni dello stantuffo e del beccuccio è $2$. Se si pone la siringa in posizione orizzontale sul bordo di un tavolo alto $1 m$, a che distanza orizzontale dal bordo del tavolo il getto colpirà il pavimento, applicando una pressione di $60 Pa$ allo stantuffo.
\end{figure}
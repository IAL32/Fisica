\begin{figure}[h!]
\textbf{Tema d'Esame di Gennaio 201}8\\ \\
Un pallavolista effettua un servizio al salto e colpisce la palla orizzontalmente. A
quale altezza minima deve colpire la palla perché questa arrivi nel campo
avversario passando a fil di rete, se la velocità del servizio è $90.0 km/h$, la rete è
alta $2.43 m$ e si trova a $9.00 m$ di distanza\\ 
\begin{boxed}

\null\hfill $h_{tot} = 3.065m$

\textbf{Soluzione:}\\
Troviamo il tempo necessario affinchè distanza percorsa dalla palla sia $9m$\\
$90km/h=\frac{90000m}{3600s}=25m/s$  \\
$v=\frac{d}{t} \qquad t=\frac{d}{v}=\frac{9m}{25m/s}=0.36s$

Troviamo la distanza percorsa sull'asse $y$ in $0.36s$ in caduta libera da fermo\\
$t=\sqrt{\frac{2\cdot h}{g}} \qquad h=\frac{v^2\cdot g}{2}= \frac{(0.36s)^2\cdot 9.81m/s^2}{2}=0.635m$\\

Sommo l'altezza della rete all'altezza percorsa in $0.36s$ \\
$h_{tot}=h_{rete} + h_{0.36} = 2.43m + 0.635m =3.065m$

\end{boxed}
\end{figure}


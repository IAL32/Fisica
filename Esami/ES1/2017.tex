\begin{figure}[h!]
\textbf{Tema d'Esame di Gennaio 2017}\\ \\
Una sferetta metallica viene lanciata verticalmente verso l’alto con modulo della velocità
$v_0= 14 m/s$  da una terrazza alta $y_0 = 22.4m$ rispetto al suolo. Si calcoli la velocità di arrivo al suolo. \\
\begin{boxed}
\textbf{Soluzione:}\\
Lanciando la pallina dalla terrazza voglio ottenere l'altezza massima raggiunta dalla pallina\\
$v=\sqrt{2\cdot g \cdot h} \qquad h=\frac{v^2}{2\cdot g}$\\
$h=\frac{(14m/s)^2}{2\cdot 9.81 m/s^2} = \frac{196m^2/s^2}{19.62 m/s^2}=9.99m \quad$ a cui va  sommato $y_0$ \\
$\quad h_{max}= 9.99m + 22.4m = 32.39m$\\ \\
Avendo ottenuto l'altezza massima basta calcolare la velocità di un corpo fermo\\
$v=\sqrt{2\cdot g\cdot h} = \sqrt{2\cdot 9.81m/s^2\cdot 33.39m}=25.21m/s$
\end{boxed}
\end{figure}



\begin{figure}[h!]
\textbf{Tema d'Esame di Febbraio 2017}\\ \\
 Un punto materiale si muove su una circonferenza di raggio $r = 1 m$ con moto
uniformemente accelerato. Al tempo $t_0 = 0$ il punto ha una velocità $v_0 = 0.1 m/s$. Dopo un tempo $t_1 = 2 s$ ha percorso uno spazio $s_1 = 40 cm$. Si calcoli il modulo dell’accelerazione a al tempo $t_2 = 4 s$

\begin{boxed}
\textbf{Soluzione:}\\
Calcolo l'accellerazione tangenziale utilizzando la formula del moto uniformemente accellerato\\
$\Delta s=v_0\cdot t_1+\frac{1}{2}\cdot a_t\cdot t_1^2$\\
$0.4m=0.1m/s\cdot 2s +\frac{1}{2}\cdot a_t \cdot 4s^2 \qquad a_t=\frac{2\cdot 0.2m}{4s^2}=0.1m/s^2$\\
Calcolo la velocità al tempo $t_2$\\
$v_1=v_0+a_t(t_1-t_0)=0.1m/s+0.2m/s=0.3m/s $\\
$v_2=v_1+a_t(t_2-t_1)=0.3m/s+0.2m/s=0.5m/s $\\
Calcolo l'accellerazione centripeta\\
$a_c=\frac{v^2}{r}=\frac{(0.5m/s)^2}{1m}=0.25m/s^2$\\
Il modulo della accellerazione è da:\\
$a_{tot}=\sqrt{a_c^2+a_t^2}=\sqrt{(0.25m/s^2)^2+(0.1m/s^2)^2}=0.27m/s^2$
\end{boxed}
\end{figure}


\begin{figure}[h!]
\textbf{Tema d'Esame di Giugno 2017}\\ \\
Un'elicottero vola orizzontalmente a $180 km/h$ e a una quota di $500 m$ lancia
un carico che deve toccare terra in un punto ben preciso. Trascurando la resistenza
dell’aria, a quale distanza orizzontale dal bersaglio l'equipaggio deve effettuare il
lancio? \\
\begin{boxed}
\textbf{Soluzione:}\\
Sapendo che $v=\frac{d}{t}$ se trovassimo il tempo impiegato per cadere dai $500m$ tramite le formule inverse ricaveremmo la distanza percorsa\\ \\
Tramite la formula del moto uniformemente accellerato scompongo la distanza percorsa sull'asse $x$:\\
$x=x_0 + v_0\cdot t + \frac{1}{2}\cdot a \cdot t^2 $\\
$500m= 0m + 0m/s\cdot t + \frac{1}{2}\cdot 9.81m/s^2\cdot t^2 \qquad$ \\
Ossia $\qquad 4.905m/s^2\cdot t^2=500m \quad t=\sqrt{101.937s^2}=10.096s $\\ \\
Quindi $d=v\cdot t=(\frac{180 000m}{3 600s})\cdot 10.096s=50m/s\cdot 10.096s=504.8m$
\end{boxed}
\end{figure}


\begin{figure}[h!]
\textbf{Tema d'Esame di Settembre 2017}\\ \\
Un punto materiale si muove su una circonferenza di raggio $r = 0.1 m$ con moto
uniformemente accelerato. Al tempo $t_0 = 0$ il punto ha una velocità $v_0 = 0.2 m/s$.
Dopo un tempo $t_1 = 1 s$ ha percorso uno spazio $s_1 = 4 cm$. Si calcoli il modulo
dell’accelerazione a al tempo $t_2 = 4 s$
\begin{boxed}
\textbf{Soluzione:}\\
Calcolo l'accellerazione tangenziale utilizzando la formula del moto uniformemente accellerato\\
$\Delta s=v_0\cdot t_1+\frac{1}{2}\cdot a_t\cdot t_1^2$\\
$0.04m=0.2m/s\cdot 1s +\frac{1}{2}\cdot a_t \cdot 1s^2 \qquad a_t=\frac{-0.16m\cdot 2}{1s^2}=-0.32m/s^2$\\
Calcolo la velocità al tempo $t_2$\\
$v_2=v_0+a_t(t_2-t_0)=0.2m/s-0.32m/s^2\cdot 4s=-1.08m/s^2 $\\
Calcolo l'accellerazione centripeta\\
$a_c=\frac{v^2}{r}=\frac{(-1.08m/s)^2}{0.1m}=11.664m/s^2$\\
Il modulo della accellerazione è da:\\
$a_{tot}=\sqrt{a_c^2+a_t^2}=\sqrt{(11.664m/s^2)^2+(-0.32m/s^2)^2}=11.668m/s^2$
\end{boxed}
\end{figure}


\begin{figure}[h!]
\textbf{Tema d'Esame di Gennaio 2016}\\ \\
Calcolare l’accelerazione centripeta di un satellite in orbita geostazionaria.
$M_{Terra}=5.972\cdot 10^{24} kg$.
\end{figure}

\begin{figure}[h!]
\textbf{Tema d'Esame di Giugno 2016}\\ \\
Sull'asse che unisce la terra con la luna (distanza terra-luna $D_{TL}=3.8\cdot 10^8 m$), a quale distanza dal centro della terra ($M_T=6.0\cdot 10^{24} kg$) la forza gravitazionale netta esercitata su un corpo di massa $M$ è nulla? (massa Luna $M_L=7.4\cdot 10^{22} kg$)
\end{figure}

\begin{figure}[h!]
\textbf{Tema d'Esame di Luglio 2016}\\ \\
Calcolare il peso in $N$ di un astronauta sulla stazione spaziale quando questa orbita ad una quota di $400 km$ sopra la superficie terrestre. La massa dell'astronauta è di $70 kg$. Il raggio della terra è $6370 km$.
\end{figure}
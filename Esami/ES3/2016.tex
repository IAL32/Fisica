\begin{figure}[h!]
\textbf{Tema d'Esame di Gennaio 2016}\\ \\
Calcolare l’accelerazione centripeta di un satellite in orbita geostazionaria.
$M_{Terra}=5.972\cdot 10^{24} kg$.
\end{figure}

\begin{figure}[h!]
\textbf{Tema d'Esame di Giugno 2016}\\ \\
Sull'asse che unisce la terra con la luna (distanza terra-luna $D_{TL}=3.8\cdot 10^8 m$), a quale distanza dal centro della terra ($M_T=6.0\cdot 10^{24} kg$) la forza gravitazionale netta esercitata su un corpo di massa $M$ è nulla? (massa Luna $M_L=7.4\cdot 10^{22} kg$)\\ \\
\noindent\fbox{
    \parbox{\textwidth}{
        \null\hfill \textbf{Soluzione:} $ F=609.18N $\\
        \textbf{Procedimento: } \\
        Trasformare tutte le unità di misura secondo le convenzioni del Sistema Internazionale.\\
        $d_{TL}=3.8\cdot 10^8m$\\ 
        $M_T=6.0 \cdot 10^{24}kg$\\
        $M_L=7.4 \cdot 10^{22}kg$\\ \\
        Sapendo che:
        \begin{gather*}
            F=F_1=F_2=G\cdot \frac{M_1\cdot M_2}{d^2}            
        \end{gather*}
        Sapendo che per esercitare forza gravitazionale nulla deve valere ($M_C$ massa del corpo che si suppone in mezzo alle due traiettorie):
        \begin{gather*}
            F_{TC}=\cancel{G}\cdot \frac{M_T\cdot \cancel{M_C}}{d_{TC}^2}=\cancel{G}\cdot \frac{M_L\cdot \cancel{M_C}}{d_{LC}^2}=F_{LC} \qquad \qquad\frac{M_T}{M_L}=\frac{d_{TC}^2}{d_{LC}^2}=\frac{6.0 \cdot 10^{24}kg}{7.4 \cdot 10^{22}kg}=81.08 
        \end{gather*}
        Ricordando che:
        \begin{gather*}
            d_{LC}^2 = (d_{TL}-d_{TC})^2     
        \end{gather*}
        Sostituendo otteniamo:
        \begin{gather*}
            d_{TC}^2=81.08 \cdot (d_{TL}-d_{TC})^2 = 81.08 \cdot (3.8 \cdot 10^8m - d_{TC})^2 \\
            d_{TC}^2=81.08 \cdot (1.44 \cdot 10^{17}m^2 + d_{TC}^2 - 7.6\cdot 10^8m \cdot d_{TC})\\ \\
            80.08\cdot d_{TC}^2 -6.162\cdot 10^{10}m \cdot d_{TC} + 1.170\cdot 10^{19}m^2 = 0\\ 
            \Delta =(6.162\cdot 10^{10})^2 -(4 \cdot 81.08 \cdot 1.170\cdot 10^{19})=2.48\cdot 10^{18}\\
            d_{TC1, TC2}= \frac{6.162\cdot 10^{10} \pm \sqrt{2.48\cdot 10^{18}}}{2\cdot 80.08}=\\
            d_{TC1}=3.94 \cdot 10^8m  \qquad d_{TC2}=3.74 \cdot 10^8m
        \end{gather*}
        
           
    }
}	
\end{figure}

\begin{figure}[h!]
\textbf{Tema d'Esame di Luglio 2016}\\ \\
Calcolare il peso in $N$ di un astronauta sulla stazione spaziale quando questa orbita ad una quota di $400 km$ sopra la superficie terrestre. La massa dell'astronauta è di $70 kg$. Il raggio della terra è $6370 km$.\\ \\

\end{figure}
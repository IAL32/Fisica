\begin{figure}[h!]
\textbf{Tema d'Esame di Gennaio 2018}\\ \\
Una persona di massa $70.0 kg$ sta su una bilancia posta all’equatore sulla
superficie di un pianeta (supposto perfettamente sferico e uniforme). Qual è il
peso misurato dalla bilancia se il diametro del pianeta è il doppio di quello della
terra, ma la sua densità media ed il suo periodo di rotazione sono gli stessi della
terra? ($M_T = 5.97\cdot 10^{24}kg, R_T = 6370 km, T_T = 24.0 h$) 
\end{figure}
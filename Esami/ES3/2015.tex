\begin{figure}[h!]
\textbf{Tema d'Esame di Gennaio 2015}\\ \\
Calcolare il valore dell'accellerazione di gravità alla superfice del pianeta Venere, sapendo che la sua velocità di fuga vale $10.36km/s$ e il raggio è di $6052km$\\ \\
    \noindent\fbox{
		\parbox{\textwidth}{
			\null\hfill \textbf{Soluzione:} $a_{gLuna}=8.86m/s^2$\\
			\textbf{Procedimento: } \\
            Trasformare tutte le unità di misura secondo le convenzioni del Sistema Internazionale.\\
            $v_{fLuna}=10.36km/s=10.36\cdot10^3m/s$\\
            $r_{luna}=6052km=6052\cdot10^3 m$\\ \\
            Sapendo che:
            \begin{gather*}
                a_g=\frac{M_{Luna}\cdot G}{r_{Luna}^2}  \qquad  \qquad v_f=\sqrt{\frac{2\cdot G \cdot M_{Luna}}{r_{luna}}}  \\
                G\cdot M_{Luna}=\frac{v_{luna}^2 \cdot r_{luna}}{2}= \frac{(10.36\cdot10^3m/s)^2 \cdot 6052\cdot10^3 m}{2}=3.2478 \cdot 10^{14}m^3/s^2\\
                a_{g_{Luna}}=\frac{G\cdot M_{Luna}}{r_{luna}^2}=\frac{3.2478 \cdot 10^{14}m^3/s^2}{(6052\cdot 10^3m)^2}=8.86m/s^2 
            \end{gather*}
            
		}
	}	
\end{figure}

\begin{figure}[h!]
\textbf{Tema d'Esame di Febbraio 2015}\\ \\
Calcolare il periodo di rotazione della Luna attorno alla Terra assumendo che percorra un'orbita circolare di raggio $384000 km$, conoscendo l'accelerazione di gravità sulla superficie della Terra, $g = 9.8 m/s^2$
 e il raggio della Terra $6370 km$. 
\end{figure}

\begin{figure}[h!]
\textbf{Tema d'Esame di Giugno 2015}\\ \\
Qual'è la velocità di fuga da un asteroide (sferico) di raggio $800km$ e per il quale l'accellerazione di gravità sulla superfice vale $6m/s^2$?
\end{figure}

\begin{figure}[h!]
\textbf{Tema d'Esame di Luglio 2015}\\ \\
Calcolare il periodo orbitale del Telescopio Spaziale Hubble (HST), che compie orbite circolari intorno alla Terra alla quota di $600km$, il raggio della Terra è $6378km$ e la massa $5.98\cdot 10^{24}kg$
\end{figure}
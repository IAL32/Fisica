\begin{figure}[h!]
\textbf{Tema d'Esame di Gennaio 2015}\\ \\
Calcolare il valore dell'accellerazione di gravità alla superfice del pianeta Venere, sapendo che la sua velocità di fuga vale $10.36km/s$ e il raggio è di $6052km$
\end{figure}

\begin{figure}[h!]
\textbf{Tema d'Esame di Febbraio 2015}\\ \\
Calcolare il periodo di rotazione della Luna attorno alla Terra assumendo che percorra un'orbita circolare di raggio $384000 km$, conoscendo l'accelerazione di gravità sulla superficie della Terra, $g = 9.8 m/s^2$
 e il raggio della Terra $6370 km$. 
\end{figure}

\begin{figure}[h!]
\textbf{Tema d'Esame di Giugno 2015}\\ \\
Qual'è la velocità di fuga da un asteroide (sferico) di raggio $800km$ e per il quale l'accellerazione di gravità sulla superfice vale $6m/s^2$?
\end{figure}

\begin{figure}[h!]
\textbf{Tema d'Esame di Luglio 2015}\\ \\
Calcolare il periodo orbitale del Telescopio Spaziale Hubble (HST), che compie orbite circolari intorno alla Terra alla quota di $600km$, il raggio della Terra è $6378km$ e la massa $5.98\cdot 10^{24}kg$
\end{figure}